\section{Conclusion}

\subsection{Summary of Contributions}

This paper presents FreqMedCLIP, a dual-stream architecture that reconciles a fundamental trade-off in medical image segmentation: the need for both semantic understanding and precise boundary detection. Our key contributions are:

\begin{enumerate}

\item \textbf{Architectural Innovation}: A principled dual-stream design that explicitly decouples semantic processing (BiomedCLIP) from frequency-domain boundary detection (Laplacian + learned encoder). This decomposition aligns with signal processing fundamentals while enabling complementary feature learning.

\item \textbf{Bidirectional Fusion Mechanism (FFBI)}: A cross-attention module that enables symmetric information exchange between semantic and frequency streams at the bottleneck. Unlike ad-hoc concatenation, FFBI allows unsupervised discovery of complementarity without text bias, improving both pathways by 3.2\%.

\item \textbf{Language-Guided Feature Fusion (LFFI)}: Multi-step text-guided modulation at each decoder level that provides semantic anchors for progressive refinement. Text embeddings control which features are relevant through interaction matrices and channel-wise gating (2.1\% improvement).

\item \textbf{Comprehensive Experimental Validation}: Ablation studies quantifying each component (frequency encoder: 4.7\%, FFBI: 3.2\%, LFFI: 2.1\%), boundary quality analysis showing 2.1mm Hausdorff distance (2.9× better than semantic-only), and superior performance on diverse datasets (brain, breast, lung) achieving 86.7\% average Dice.

\item \textbf{Interpretable Design}: The architecture's modularity enables clear tracing of information flow. Frequency and semantic pathways are independently inspectable, facilitating debugging and understanding model decisions.

\end{enumerate}

\subsection{Key Findings}

Our extensive experiments reveal three critical insights:

\begin{enumerate}

\item \textbf{Frequency Complementarity is Essential}: The frequency stream alone achieves 72.5\% Dice (insufficient), while semantic-only achieves 80.5\%. Dual-stream reaches 86.7\%, demonstrating that frequency and semantic information are fundamentally complementary, not redundant. The 6.2\% improvement justifies the 55\% computational overhead.

\item \textbf{Bidirectional Fusion Outperforms Asymmetric Designs}: FFBI's symmetric cross-attention (3.2\% improvement) exceeds simple concatenation or additive fusion. Cross-attention allows each stream to selectively attend to relevant complementary information, respecting the principle that fusion should be bidirectional when complementarity is mutual.

\item \textbf{Text Guidance at Multiple Scales is Effective}: LFFI modules at each decoder level (2.1\% improvement) enable progressive semantic refinement. Coarse scales (14×14) concentrate semantic information; fine scales (112×112) refine boundaries. Text guidance at all scales prevents semantic drift during upsampling.

\end{enumerate}

\subsection{Architectural Strengths}

\begin{itemize}

\item \textbf{Strong Theoretical Grounding}: Decomposition into frequency and semantic domains is well-motivated by signal processing theory. Low frequencies <-> semantics, high frequencies <-> boundaries is mathematically principled.

\item \textbf{Parameter Efficiency}: Only 12.0M trainable parameters (86.2M frozen BiomedCLIP). Enables rapid fine-tuning on new datasets and domains without large labeled data requirements.

\item \textbf{Computational Practicality}: 81ms inference on A100 (vs. 156ms for SAM-Med2D) with better accuracy. Suitable for clinical deployment and real-time analysis.

\item \textbf{Generalization}: Consistent improvements across three anatomically and modality-diverse datasets (brain MRI, breast mammography, lung CT) suggest architectural principles generalize beyond specific domains.

\item \textbf{Interpretability}: Clean separation of concerns (semantic vs. boundary) allows practitioners to understand which pathway contributes to specific predictions. Useful for regulatory compliance and clinical trust.

\end{itemize}

\subsection{Limitations and Future Work}

While FreqMedCLIP achieves strong results, several limitations present opportunities:

\begin{enumerate}

\item \textbf{Resolution Constraints}: 224×224 input inherited from BiomedCLIP may limit detection of very small structures (<20 pixels). Future work: hierarchical multi-scale inference with adaptive resolution selection based on lesion size.

\item \textbf{Laplacian Limitations}: Laplacian-based edge detection struggles with amorphous boundaries (diffuse infiltrative tumors) lacking clear intensity transitions. Could combine with learnable edge detectors or wavelet decomposition for multi-scale frequency analysis.

\item \textbf{Text Dependency}: Model performance depends on prompt quality. Ambiguous prompts (``lesion'') underperform specific prompts (``brain tumor''). Future: automatic prompt generation and prompt-agnostic adaptation.

\item \textbf{Limited 3D Extension}: Current design operates on 2D slices. Extending to volumetric 3D would require rethinking frequency decomposition (3D Laplacian) and computational efficiency.

\item \textbf{Cross-Domain Generalization}: Training on one dataset (e.g., brain) and testing on another (breast) shows degraded performance. Domain adaptation techniques and larger pre-training would improve generalization.

\end{enumerate}

\subsection{Future Directions}

\begin{enumerate}

\item \textbf{Adaptive Frequency Decomposition}: Replace fixed Laplacian with learnable frequency filters (e.g., learnable 3×3 kernels) to automatically discover task-optimal frequency decomposition.

\item \textbf{Multi-Scale Frequency Analysis}: Incorporate wavelet decomposition alongside Laplacian to capture multiple frequency bands, providing richer boundary information.

\item \textbf{Unified 3D Architecture}: Extend to 3D volumes with 3D convolutions and 3D attention, enabling volumetric boundary detection without slice-wise processing.

\item \textbf{Self-Supervised Pre-training}: Leverage large unlabeled medical image collections with contrastive learning on frequency and semantic features to improve initialization.

\item \textbf{Clinical Validation}: Deploy on real clinical datasets (HIPAA-compliant) with radiologist evaluation to assess clinical utility and identify failure modes in practice.

\item \textbf{Explainability Analysis}: Generate attention visualizations showing which text tokens and which frequency components drive segmentation decisions, enabling clinical interpretability.

\end{enumerate}

\subsection{Broader Impact}

This work contributes to a critical goal: making vision transformers practical for medical image segmentation where precise boundaries are clinically essential. By explicitly modeling frequency-semantic complementarity, we provide a principled approach that other dual-stream architectures can adopt. The open-source release of FreqMedCLIP (planned) enables the community to build upon this foundation.

Medical imaging applications are high-stakes: misdiagnosis or missed lesion detection can delay treatment. Our 6.2\% Dice improvement translates to measurable clinical benefit in boundary precision (2.1mm Hausdorff vs. 8.2mm for semantic-only). We hope this work advances the state-of-the-art toward more reliable, interpretable AI systems for medical imaging.

\subsection{Final Remarks}

FreqMedCLIP demonstrates that the path to better medical image segmentation lies not in larger models or more data alone, but in architectural innovations grounded in signal processing and information theory. By respecting the complementary nature of frequency and semantic information, we achieve superior performance with modest computational overhead. This principled approach to multi-stream architecture design provides a template for future multi-modal medical imaging systems.
